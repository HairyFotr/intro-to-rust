\section{Memory Safety in Rust}

%%%

\begin{frame}{Three Key Promises}

To guarantee memory safety, Rust gives us three key promises:

\begin{itemize}
	\item No null pointer dereferences
		\begin{itemize}
			\item There are no null pointers in safe Rust
			\item For error handling and control flow, \texttt{Option} and
				\texttt{Result} types are used.
		\end{itemize}
	\pause
	\item No dangling pointers
		\begin{itemize}
			\item The concepts of "ownership", "borrowing" and "lifetimes" prevent the
				use of uninitialized or freed pointers
		\end{itemize}
	\pause
	\item No buffer overruns
		\begin{itemize}
		\item There's no pointer arithmetic in safe Rust
		\item Arrays in Rust are not just pointers
		\item There are runtime bounds checks for indexing
		\item But most stdlib functions use iterators, which are checked at
			compile time
		\end{itemize}
\end{itemize}

\end{frame}


%%%

\begin{frame}[fragile]{Ergonomic scaffolding with \texttt{unwrap()}}
Explicit error handling may seem like a huge barrier.

Extracting values from \texttt{Option<T>}, \texttt{Result<T, E>} and may others fast with \texttt{unwrap()} and abort the program on failure:

\begin{minted}{rust}
fn main() {
    // Assuming `Cargo.toml` will allways be here
    // TODO: handle error late in development
    let buffer = some_io_function("Cargo.toml").unwrap();
    // ...
}

fn some_io_function(path: String) -> io::Result<String> {
    // This may fail
}
\end{minted}

\end{frame}

%%%

\subsection{Promise 1: No null pointer dereferences}

%%%

\begin{frame}{Promise 1: No null pointer dereferences}

\begin{block}{Null pointers are useful.}
They can indicate the absence of optional information.\\
They can indicate failures.\\
\pause
But they can introduce severe bugs.
\end{block}
\vspace{1em}
\pause
\begin{block}{Rust separates the concept of a pointer from the concept of an\\
		optional or error value.}
	Optional values are handled by \texttt{Option<T>}.\\
  Error values are handled by \texttt{Result<T, E>}.\\
	Many helpful tools to do error handling.
\end{block}

\end{frame}

%%%

\begin{frame}[fragile]{You already saw \texttt{Option<T>}}
\begin{minted}{rust}
fn safe_div(n: i32, d: i32) -> Option<i32> {
    if d == 0 {
        return None;
    }
    Some(n / d)
}
\end{minted}
\end{frame}

%%%

\begin{frame}[fragile]{There's also \texttt{Result<T, E>}}
\begin{minted}{rust}
enum Result<T, E> {
    Ok(T),
    Err(E)
}
\end{minted}
\end{frame}

%%%

\begin{frame}[fragile]{How to use \texttt{Result}s:}
\begin{minted}{rust}
enum Error {
    DivisionByZero,
}

fn safe_div(n: i32, d: i32) -> Result<i32, Error> {
    if d == 0 {
        return Err(Error::DivisionByZero);
    }
    Ok(n / d)
}
\end{minted}
\end{frame}

%%%

\begin{frame}[fragile]{But \texttt{Result} can get tedious...}
\begin{minted}{rust}
fn do_calc() -> Result<i32, String> {
    let a = match do_subcalc1() {
        Ok(val) => val,
        Err(msg) => return Err(msg),
    }
    let b = match do_subcalc2() {
        Ok(val) => val,
        Err(msg) => return Err(msg),
    }
    Ok(a + b)
}
\end{minted}
\end{frame}

%%%

\begin{frame}[fragile]{Ergonomic error handling with the \texttt{try!} macro}
\begin{minted}{rust}
fn do_calc() -> Result<i32, String> {
    let a = try!(do_subcalc1());
    let b = try!(do_subcalc2());
    Ok(a + b)
}
\end{minted}
\end{frame}

%%%

\begin{frame}[fragile]{Mapping Errors}
\begin{minted}{rust}
fn do_subcalc() -> Result<i32, String> { ... }
fn do_calc() -> Result<i32, Error> {
    let res = do_subcalc();
    let mapped = res.map_err(|msg| {
        println!("Error: {}", msg);
        Error::CalcFailed
    });
    let val = try!(mapped);
    Ok(val + 1)
}
\end{minted}
\end{frame}

%%%

\begin{frame}[fragile]{Mapping Errors: A closer look}
\begin{minted}{rust}
let mapped = res.map_err(|msg| Error::CalcFailed);
\end{minted}
...is the same as...
\begin{minted}{rust}
let mapped = match res {
    Ok(val) => Ok(val),
    Err(msg) => Err(Error::CalcFailed),
}
\end{minted}
\end{frame}


\subsection{Promise 2: No dangling pointers}

%%%

\begin{frame}{Promise 2: No dangling pointers}

\begin{itemize}
	\item Rust programs never try to access a heap-allocated value after it has
		been freed.
	\item By default, no garbage collection or reference counting involved!
	\item Everything is enforced at compile-time.
\end{itemize}

\end{frame}

%%%

\begin{frame}[label={threerules}]{Three Rules}

\begin{block}{Rule 1}
Every value has a single owner at any given time.
\end{block}
\pause
\begin{block}{Rule 2}
You can borrow a reference to a value, so long as the
reference doesn’t outlive the value.
\end{block}
\pause
\begin{block}{Rule 3}
You can only modify a value when you have exclusive
access to it.
\end{block}

\end{frame}

%%%

\begin{frame}{Ownership}

\begin{itemize}
	\item Variable bindings own their values
	\item A struct owns its fields
	\item An enum owns its values
	\item Every heap-allocated value has a single pointer that owns it
	\item All values are dropped when their owner is dropped
\end{itemize}

\end{frame}

%%%

\begin{frame}[fragile]{Ownership: Scoping}

If a value goes out of scope, the corresponding memory is automatically freed.

\begin{minted}{rust}
{
    let s: String = "Chuchichästli".to_string();
} // s goes out of scope, memory is freed
\end{minted}

\end{frame}

%%%

\begin{frame}[fragile]{Ownership: Move Semantics}

Ownership is moved by default.

\begin{minted}{rust}
let s: String = "Chuchichästli".to_string();

// t1 takes ownership from s
let t1 = s;

// compile-time error: use of moved value s
let t2 = s;
\end{minted}

\end{frame}

%%%

\begin{frame}[fragile]{Ownership: Opt-in Implicit Copy Semantics}

Types that implement the \texttt{Copy} marker trait (more about traits later)
are copied instead of moved. The stdlib implements \texttt{Copy} for all
primitive types.

\begin{minted}{rust}
let pi = 3.1415926f32;
let foo = pi;
let bar = pi; // This is fine!
\end{minted}

\end{frame}

%%%

\begin{frame}[fragile]{Ownership: Opt-in Explicit Copy Semantics}

If you prefer copies to be explicit, you can implement the \texttt{Clone} trait
instead.

\begin{minted}{rust}
let s = "Chuchichästli".to_string();
let t1 = s.clone();
let t2 = s.clone();
\end{minted}

\end{frame}

%%%

\begin{frame}[fragile]{Ownership: Deriving Copy / Clone}

The compiler can automatically derive implementations of \texttt{Copy} and
\texttt{Clone} for us.

\begin{minted}{rust}
#[derive(Copy, Clone)]
struct Color {
    r: u8,
    g: u8,
    b: u8
}
\end{minted}

\end{frame}

%%%

\begin{frame}[fragile]{Ownership: Function Parameters}

But what about this?

\begin{minted}{rust}
fn print_loud(text: String) { println!("{}!!!!!", text); }
let s = "Hello, Hackers".to_string();
print_loud(s);
println!("{}", s);
\end{minted}

\pause
\sep

\begin{minted}[fontsize=\footnotesize]{text}
error: use of moved value: `s`
println!(“{}”, s);
               ^
note: `s` moved here because it has type `collections::string::String`,
which is non-copyable
print_loud(s);
           ^
\end{minted}

\end{frame}

%%%

\begin{frame}[fragile]{Borrowing}

Instead of moving a value, it can also be borrowed.

\begin{minted}{rust}
fn print_loud(text: &String) { println!("{}!!!!!", text); }
let s = "Hello, Hackers".to_string();
print_loud(&s);
println!("Original value was {}", s);
\end{minted}

Many functions can borrow at the same time, because they cannot modify.

\end{frame}

%%%

\begin{frame}[fragile]{Mutable Borrowing}

If you need exclusive (=write) access, you can use mutable borrows.

\begin{minted}{rust}
fn make_loud(text: &mut String) { text.push_str("!!!!!"); };
let mut s = "Hello, Hackers".to_string();
make_loud(&mut s);
println!("New value is {}", s);
\end{minted}

While borrow a mutable reference to a value, that reference is the only way to
access that value at all.
\end{frame}

%%%

\begin{frame}[fragile]{Borrowing prevents moving}

While borrowed, a move must be prevented. Otherwise you might end up with a
dangling pointer.

\begin{minted}{rust}
let x = String::new();
let borrow = &x;
let y = x;
\end{minted}

\sep

\begin{minted}[fontsize=\footnotesize]{text}
error: cannot move out of `x` because it is borrowed [E0505]
    let y = x;
            ^
note: borrow of `x` occurs here
    let borrow = &x;
                  ^
\end{minted}

\end{frame}

%%%

\begin{frame}[fragile]{Lifetimes}

What's the problem here?

\begin{minted}{rust}
let borrow;
let x = String::new();
borrow = &x;
\end{minted}

\sep

\begin{minted}[fontsize=\footnotesize]{text}
error: `x` does not live long enough
    borrow = &x;
              ^
\end{minted}

\end{frame}

%%%

\begin{frame}[fragile]{Lifetimes}

The lifetime of the borrow is longer than the lifetime of `x`.

\begin{minted}[fontsize=\footnotesize]{rust}
let borrow;
let x = String::new();
borrow = &x;
\end{minted}

This can also be visualized differently:

\begin{minted}[fontsize=\footnotesize]{rust}
{
    let borrow;
    {
        let x = String::new();
        borrow = &x;
    }
}
\end{minted}

Using lifetime checking, the compiler guarantees that there are no dangling
pointers.

\end{frame}

\subsection{Promise 3: No buffer overruns}

\begin{frame}{No buffer overruns: Recap}
\begin{itemize}
	\item There's no pointer arithmetic in safe Rust
	\item Arrays in Rust are not just pointers
	\item There are runtime bounds checks for indexing
	\item But most stdlib functions use iterators, which are checked at
		compile time
\end{itemize}
\end{frame}


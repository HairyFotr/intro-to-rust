\section{Reading Rust}

%%%

\begin{frame}[fragile]{Basic Syntax}
\begin{alltt}
\hi{fn gcd}(mut n: u64, mut m: u64) \hi{->} u64 \{
    assert!(n != 0 && m != 0);
    while m != 0 \{
        if m < n \{
            let t = m; m = n; n = t;
        \}
        m = m % n;
    \}
    n
\}
\end{alltt}
\end{frame}

%%%

\begin{frame}[fragile]{Basic Syntax}
\begin{alltt}
fn gcd(\hi{mut} n: u64, \hi{mut} m: u64) -> u64 \{
    assert!(n != 0 && m != 0);
    while m != 0 \{
        if m < n \{
            let t = m; m = n; n = t;
        \}
        m = m % n;
    \}
    n
\}
\end{alltt}
\end{frame}

%%%

\begin{frame}[fragile]{Basic Syntax}
\begin{alltt}
fn gcd(mut \hi{n: u64}, mut \hi{m: u64}) -> u64 \{
    assert!(n != 0 && m != 0);
    while m != 0 \{
        if m < n \{
            let t = m; m = n; n = t;
        \}
        m = m % n;
    \}
    n
\}
\end{alltt}
\end{frame}

%%%

\begin{frame}[fragile]{Basic Syntax}
\begin{alltt}
fn gcd(mut n: u64, mut m: u64) -> u64 \{
    \hi{assert!}(n != 0 && m != 0);
    while m != 0 \{
        if m < n \{
            let t = m; m = n; n = t;
        \}
        m = m % n;
    \}
    n
\}
\end{alltt}
\end{frame}

%%%

\begin{frame}[fragile]{Basic Syntax}
\begin{alltt}
fn gcd(mut n: u64, mut m: u64) -> u64 \{
    assert!(n != \hi{0} && m != \hi{0});
    while m != \hi{0} \{
        if m < n \{
            let t = m; m = n; n = t;
        \}
        m = m % n;
    \}
    n
\}
\end{alltt}
\end{frame}

%%%

\begin{frame}[fragile]{Basic Syntax}
\begin{alltt}
fn gcd(mut n: u64, mut m: u64) -> u64 \{
    assert!(n != 0 && m != 0);
    while m != 0 \{
        if m < n \{
            \hi{let t = m}; m = n; n = t;
        \}
        m = m % n;
    \}
    n
\}
\end{alltt}
\end{frame}

%%%

\begin{frame}[fragile]{Basic Syntax}
\begin{alltt}
fn gcd(mut n: u64, mut m: u64) -> u64 \{
    assert!(n != 0 && m != 0);
    \hi{while m != 0 \{}
        \hi{if m < n \{}
            let t = m; m = n; n = t;
        \hi{\}}
        m = m % n;
    \hi{\}}
    n
\}
\end{alltt}
\end{frame}

%%%

\begin{frame}[fragile]{Basic Syntax}
\begin{alltt}
fn gcd(mut n: u64, mut m: u64) -> u64 \{
    assert!(n != 0 && m != 0);
    while m != 0 \{
        if m < n \{
            let t = m; m = n; n = t;
        \}
        m = m % n;
    \}
    \hi{n}
\}
\end{alltt}
\end{frame}

%%%

\begin{frame}[fragile]{Structs}
\pause\begin{minted}{rust}
pub struct Point {
  x: f64,
  y: f64,
}
\end{minted}
\pause
\begin{minted}{rust}
impl Point {
  /// static method ("constructor" per convention)
  pub fn new() -> Point {
    Point { x: 0.0, y: 0.0 }
  }
  pub fn distance_from_zero(&self) -> f64 {
    (self.x*self.x + self.y*self.y).sqrt()
  }
}
\end{minted}
\end{frame}

\begin{frame}[fragile]{Structs: Usage}
\begin{minted}{rust}
let p = Point::new();
assert_eq!(p.x, 0.0);
assert_eq!(p.y, 0.0);
\end{minted}
\begin{minted}{rust}
let p = Point { x: 1.0, y: 2.0 };
assert_eq!(p.x, 1.0);
assert_eq!(p.y, 2.0);
\end{minted}
\begin{minted}{rust}
let p = Point { x: 3.0, y: 4.0 };
assert_eq!(p.distance_from_zero(), 5.0);
\end{minted}
\end{frame}

\begin{frame}[fragile]{Enumerations}
\begin{columns}[t,onlytextwidth]
\column{.35\textwidth}
\begin{minted}{rust}
pub enum Color {
  Red,
  Green,
  Blue,
}
\end{minted}
\column{.65\textwidth}
\begin{minted}{rust}
let color = Color::Red;
match color {
  Color::Red => println!("Got red"),
  Color::Green => println!("Got green"),
  Color::Blue => println!("Got blue"),
}
\end{minted}
\end{columns}
  \begin{itemize}
    \item<1-> In its simplest form similar to C/C++ (but type safe)
	\item<1-> Data that is one of several possible variants
    \item<2> We can match (switch) on enums
    \item<2> We \emph{must} match all variants
  \end{itemize}
\end{frame}

\begin{frame}[fragile]{Enums with data}
\begin{columns}[t,onlytextwidth]
\column{.37\textwidth}
\begin{minted}{rust}
pub enum Shape {
  Circle{radius: f64},
  Square{side: f64},
}
\end{minted}
\column{.63\textwidth}
\pause\begin{minted}{rust}
let shape = Shape::Circle{radius: 1.0};
match shape {
  Shape::Circle{radius} => {
    println!("Circle with radius {}", radius)
  }
  _ => panic!("Must be a circle!"),
}
\end{minted}
\end{columns}
\begin{itemize}
  \item<1-> A \texttt{Shape} is Either a \texttt{Circle} with \texttt{radius} or a
    \texttt{Square} with \texttt{side}
  \item<2-> We can use match to "extract" data
\end{itemize}
\end{frame}


%%%

\begin{frame}[fragile]{Traits}
\begin{minted}{rust}
pub trait HasArea {
  fn area(&self) -> f64;
}
\end{minted}
\begin{itemize}
	\item Like interfaces
	\item Defines functionality a type must provide
\end{itemize}
\end{frame}

%%%

\begin{frame}[fragile]{Trait Implementation}
\begin{minted}{rust}
pub struct Circle {
  x: f64,
  y: f64,
  radius: f64,
}

impl HasArea for Circle {
  fn area(&self) -> f64 {
    std::f64::consts::PI * (self.radius * self.radius)
  }
}
\end{minted}
\end{frame}

\begin{frame}[fragile]{Trait Implementation}
\begin{minted}{rust}
pub struct Square {
  x: f64,
  y: f64,
  side: f64,
}

impl HasArea for Square {
  fn area(&self) -> f64 {
    self.side * self.side
  }
}
\end{minted}
\end{frame}

%%%

\begin{frame}[fragile]{Trait Based Generics}

\begin{minted}{rust}
pub fn print_area<Shape: HasArea>(shape: &Shape) {
  println!("{}", shape.area());
}
\end{minted}
\pause
\begin{minted}{rust}
...

print_area(&Circle{x: 1.0, y: 2.0, radius: 3.0}); // Shape is Circle
print_area(&Square{x: 4.0, y: 5.0, side: 5.0}); // Shape is Square
\end{minted}

\end{frame}

%%%

\begin{frame}[fragile]{Generic Types}
\begin{minted}{rust}
struct Coords<T> {
  x: T,
  y: T,
}
\end{minted}
\pause
\begin{minted}{rust}
...

Coords { x: 200, y: 800 }  // OK
Coords { x: 1.3, y: 4.7 }  // Also OK
\end{minted}
\end{frame}

%%%

\begin{frame}[fragile]{ Generic Enumerations}
\begin{minted}{rust}
enum Option<T> {
  Some(T),
  None
}
\end{minted}
\end{frame}

%%%

\begin{frame}[fragile]{Application of Option<T>}
\begin{minted}{rust}
fn safe_div(n: i32, d: i32) -> Option<i32> {
  if d == 0 {
    return None;
  }
  Some(n / d)
}
\end{minted}
\end{frame}

%%%

\begin{frame}[fragile]{Matching an Option}
\begin{minted}{rust}
match safe_div(num, denom) {
  None => println!("No quotient."),
  Some(v) => println!("Quotient is {}.", v)
}
\end{minted}
\end{frame}

%%%


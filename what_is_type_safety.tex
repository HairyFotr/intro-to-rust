\section{What is Type Safety?}

%%%

\begin{frame}[fragile]{A C Program}
\begin{minted}{c}
int main(int argc, char **argv) {
    unsigned long a[1];
    a[3] = 0x7ffff7b36cebUL;
    return 0;
}
\end{minted}
	According to C99, undefined behavior. Output:

	{\footnotesize \tt undef: Error: .netrc file is readable by others.\\
	undef: Remove password or make file unreadable by others.}
\end{frame}

%%%

\begin{frame}{Definitions}
	\begin{itemize}
		\item If a program has been written so that no possible execution can
			exhibit undefined behavior, we say that program is \textbf{well defined}.
		\item If a language’s type system ensures that every program is well
			defined, we say that language is \textbf{type safe}.
	\end{itemize}
\end{frame}

%%%

\begin{frame}[fragile]{Type Safe Languages}
	\begin{itemize}
		\item C and C++ are not type safe.
		\item Python is type safe:
\begin{minted}[fontsize=\footnotesize]{python}
>>> a = [0]
>>> a[3] = 0x7ffff7b36ceb
Traceback (most recent call last):
File "", line 1, in <module>
IndexError: list assignment index out of range
>>>
\end{minted}
		\item Java, JavaScript, Ruby, and Haskell are also type safe.
	\end{itemize}
\end{frame}

%%%

\begin{frame}{It's Ironic.}
	\begin{itemize}
		\item C and C++ are not type safe.
		\item Yet they are being used to implement the foundations of a system.
		\item Rust tries to resolve that tension
	\end{itemize}
\end{frame}
